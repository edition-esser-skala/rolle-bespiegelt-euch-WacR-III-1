\documentclass[tocstyle=ref-genre]{ees}

\shorttitle{Bespiegelt euch}

\begin{document}

\eesTitlePage

\eesCriticalReport{
  1.1  & 3     & vl 2    & 3rd \quarterNote\ in \B1: e′8–d′8 \\
       & 16    & vl 1    & 1st \sixteenthNoteDotted\ in \B1: d′16. \\
       & 16    & vla     & 8th \thirtysecondNote\ in \B1: a32 \\
       & 19    & ob 1, vl 1 & 3rd \quarterNote\ in \B1: g′8.–\sharp f′32–g′32 \\
       & 24    & vl 1    & 4th \eighthNote\ in \B1: \sharp f″8 \\
       & 70    & vl 2    & 2nd \eighthNote\ in \B1: d″8 \\
       & 74    & T       & 1st \eighthNote\ in \B1: \quaverRest \\
       & 78    & vl 1    & 1st \sixteenthNoteDotted\ in \B1: d′16. \\
       & 80    & vl 1, 2 & 1st \quarterNote\ in \B1: d′8–b8 \\
       & 110   & vl 2    & 1st \eighthNote\ in \B1: \quaverRest \\
  1.4  & 15–21 & org     & rests in \B1 \\
       & 18    & S       & 1st \quarterNote\ in \B1: \flat e″4 \\
       & 20    & vla     & last \eighthNote\ in \B1: a8 \\
       & 34    & A       & 1st \halfNote\ in \B1: a′2 \\
       & 47    & B       & 1st \halfNote\ in \B1: f2 \\
  1.6  & 8     & T       & last \quarterNote\ in \B1: f′4 \\
  1.8  & 20    & vl 2    & 5th \eighthNote\ in \B1: a′8 \\
       & 35    & vl 2    & 3rd \eighthNote\ in \B1: \sharp d″8 \\
  1.11 & 1     & Ev      & 8th \sixteenthNote\ in \B1: \sharp f′16 \\
  1.13 & 16    & org     & 2nd \halfNote\ in \B1: g2 \\
  1.14 & –     & org     & In \B1, bass figures only appear in bars
                           2, 5, 6, 15, 20, 21, 53, 54, 91, 113–115,
                           124, 128, and 131–138. \\
       & 99    & vla     & bar in \B1: a4–\quaverRest \\
       & 99    & org     & bar in \B1: A4–\quaverRest \\
       & 120   & vl 1    & 1st \eighthNote\ in \B1: \sharp d″8 \\
  1.15 & 3     & Ev      & 4th \eighthNote\ in \B1: \flat e′8 \\
       & 11    & org     & 1st \quarterNote\ in \B1: \flat B4 \\
       & 54    & vl 2    & last \quarterNote\ in \B1: g′16.–g′32–g′16.–g′32 \\
       & 57    & vla     & 2nd note in \B1: f′4 \\
  1.16 & –     & org     & In \B1, bass figures only appear in bars 1–5,
                           9–11, 46–48, and 51. \\
       & 20    & T       & 1st \sixteenthNote\ in \B1: c′16 \\
       & 24    & vla     & 1st \quarterNote\ in \B1: \flat e4 \\
       & 38    & vla     & 2nd \eighthNote\ in \B1: g8 \\
       & 41    & vla     & 2nd/3rd \eighthNote\ in \B1: a8–f′8 \\
       & 45    & org     & 3rd \eighthNote\ in \B1: d8 \\
       & 63    & fag 1   & 1st \eighthNote\ in \B1: g16–c′16 \\
  1.17 & 1     & B       & 2nd \quarterNote\ in \B1: d4 \\
  1.18 & 15    & Ev      & 3rd \quarterNote\ in \B1: c′8–c′16–d′16 \\
       & 25    & vla     & 2nd/3rd \eighthNote\ in \B1: e′8–c′8 \\
       & 25    & T       & 2nd \eighthNote\ in \B1: \sharp d′8 \\
  1.19 & 9     & vl 2, A & 3rd \quarterNote\ in \B1: \sharp f′8–\sharp f′8 \\
  1.20 & –     & org     & In \B1, bass figures only appear
                           in bars 14, 82, and 83. \\
       & 27    & vl 1    & 2nd \quarterNote\ in \B1: a″16–\sharp g″16 \\
       & 32    & A       & grace note missing in \B1 \\
       & 58    & A       & grace note missing in \B1 \\
       & 86    & cor 2   & 1st \eighthNote\ in \B1: b′8 \\
       & 97    & cor 2   & last \eighthNote\ in \B1: b′8 \\
       & 103   & org     & 2nd \quarterNote\ in \B1: \sharp f4 \\
  \midrule
  2.1  & 1–6   & org     & bars missing in \B1 \\
  2.4  & 2     & vl 2    & 3rd \eighthNote\ in \B1: b′8 \\
       & 7     & A       & grace note missing in \B1 \\
       & 14    & vla     & last \eighthNote\ in \B1: \sharp g′8 \\
  2.6  & 6     & S       & last \quarterNote\ in \B1: b′4 \\
  2.7  & 12    & Pi      & last \quarterNote\ in \B1: b4 \\
  2.10 & 10    & S       & last \eighthNote\ in \B1: a′8 \\
       & 10    & A       & grace note missing in \B1 \\
       & 11    & vla     & 1st \eighthNote\ in \B1: d′4 \\
       & 12    & T       & last \sixteenthNote\ in \B1: \flat e′16 \\
       & 13    & vl 2    & 1st \quarterNote\ in \B1: g′8–g′8 \\
       & 25    & vl, vla, S & in \B1 duplicate of bar 26 \\
       & 32    & T       & 3rd \quarterNote\ in \B1: \flat e′4 \\
       & 35    & S       & 1st \quarterNote\ in \B1: a′4 \\
  2.12 & –     & org     & In \B1, bass figures only appear in bars
                           10, 29, and 33. \\
       & 4     & fl 2, S 2 & grace note missing in \B1 \\
       & 10    & vl 2, S, A & grace note missing in \B1 \\
       & 12    & vl 2, S, A & grace note missing in \B1 \\
       & 20    & fl 2    & grace note missing in \B1 \\
  2.13 & 7     & org     & 1st \quarterNote\ in \B1: \flat e8.–\flat e16 \\
       & 13    & vla     & 1st \halfNote\ in \B1: d′2 \\
       & 19    & vla     & 3rd \eighthNote\ in \B1: \semiquaverRest–f16 \\
       & 19    & org     & 3rd \eighthNote\ in \B1: \semiquaverRest–F16 \\
  2.14 & –     & org     & In \B1, bass figures only appear in bars 1 to 10, 16,
                           25 to 29, 33, 49, 83, 84, 88 to 93, 99, and 102. \\
       & 11    & vla     & 6th \sixteenthNote\ in \B1: f′16 \\
       & 11    & org     & 6th \sixteenthNote\ in \B1: f16 \\
       & 13    & vla     & 6th \sixteenthNote\ in \B1: \flat b′16 \\
       & 13    & org     & 6th \sixteenthNote\ in \B1: \flat b16 \\
       & 23    & org     & 1st \quarterNote\ in \B1: f4 \\
       & 58    & org     & 6th \sixteenthNote\ in \B1: a16 \\
  2.19 & 10    & Ev      & 3rd \quarterNote\ in \B1: a′4 \\
  2.20 & 10    & A       & 1st \halfNote\ in \B1: g′2 \\
  2.22 & –     & org     & bass figures missing in \B1 \\
       & 36    & vl 1    & 1st \eighthNote\ in \B1: \sharp f′8 \\
  2.23 & –     & org     & bass figures missing in \B1 \\
  2.24 & –     & org     & bass figures missing in \B1 \\
       & 7     & T       & last \quarterNote\ in \B1: d′4 \\
}

\eesToc{
\textit{Soloists}\\[1ex]
Magd I–III (Soprano)\\
Evangelist, Petrus, Uebelthäter II (Tenore)\\
Hauptmann, Jesus, Pilatus, Uebelthäter I (Basso)
\par\bigskip
\part{part1}

\begin{movement}{bespiegelteuch}
  \voice[Chor]
  Beſpiegelt euch in Jeſu Leiden,\\
  hier iſt der Urſprung wahrer Freuden,\\
  hier öffnet ſich das Paradies.\\
  Zerbrechet Babels Schaugerüſte,\\
  verlaßt das Blendwerck eitler Lüſte,\\
  umfaßt den Stamm,\\
  an dem das Lamm\\
  ſich vor die Sünder martern ließ.
\end{movement}

\begin{movement}{gutenacht}
  \voice[Chor]
  Gute Nacht, o Weſen,\\
  das die Welt erleſen,\\
  mir gefällſt du nicht.\\
  Gute Nacht, ihr Sünden,\\
  bleibet weit dahinden,\\
  kommt nicht mehr ans Licht.\\
  Gute Nacht, du Stolz und Pracht,\\
  dir ſey gantz, du Laſterleben,\\
  gute Nacht gegeben.
\end{movement}

\begin{movement}{jesusginghinauf}
  \voice[Evangelist]
  Und Jeſus ging hinaus\\
  nach Seiner Gewohnheit an den Öhlberg.\\
  Es folgeten Ihm aber Seine Jünger nach\\
  an den ſelbgen Ort.
\end{movement}

\clearpage
\begin{movement}{lassetunszu}
  \voice[Chor]
  Laßet uns zu Jeſu hinaus gehen\\
  außer dem Lager\\
  und Seine Schmach tragen.\\
  (\bibleverse{Heb}(13:13))
\end{movement}

\begin{movement}{alserdahinkam}
  \voice[Evangelist]
  Und als Er dahin kam,\\
  ſprach Er zu ihnen:

  \voice[Jesus]
  Betet, auf daß ihr nicht in Anfechtung fallet.

  \voice[Evangelist]
  Und Er riß sich von ihnen\\
  bey einen Steinwurff\\
  und kniete nieder, betete und ſprach:

  \voice[Jesus]
  Vater, willſt du, ſo nimm dieſen Kelch von mir.\\
  Doch nicht mein, ſondern dein Wille geſchehe.

  \voice[Evangelist]
  Es erſchien Ihm aber ein Engel vom Himmel\\
  und ſtärckte Ihn.
\end{movement}

\begin{movement}{deinenengel}
  \voice[Chor]
  Deinen Engel zu mir ſende,\\
  der des böſen Feindes Macht,\\
  Liſt und Anſchlag von mir wende\\
  und mich halt in guter Acht,\\
  der auch endlich mich zur Ruh\\
  trage nach dem Himmel zu.
\end{movement}

\begin{movement}{undeskam}
  \voice[Evangelist]
  Und es kam, daß Er mit dem Tode rang,\\
  und betete heftiger.\\
  Es ward aber Sein Schweiß wie Blutstropfen,\\
  die fielen auf die Erde.
\end{movement}

\begin{movement}{wiequaelet}
  \voice[Soprano]
  Wie quälet ſich das höchſte Guth,\\
  mein treuer Heiland ſchwitzet Blut,\\
  Sein Kummer macht mir Angſt und Schrecken.\\
  Doch nein, mein Hertz, erhole dich:\\
  Der Liebe Brunnquell öffnet ſich\\
  und giebt ſich fließend, dir zu ſchmecken.
\end{movement}

\begin{movement}{understund}
  \voice[Evangelist]
  Und Er ſtund auf von dem Gebet\\
  und kam zu Seinen Jüngern,\\
  und fand ſie ſchlafend vor Traurigkeit,\\
  und ſprach zu ihnen:

  \voice[Jesus]
  Was ſchlafet ihr? Stehet auf und betet,\\
  auf daß ihr nicht in Anfechtung fallet.

  \voice[Evangelist]
  Da Er aber noch redete, ſiehe, die Schaar,\\
  und einer von den Zwölfen, genannt Judas,\\
  gang für ihnen her und nahete ſich zu Jeſu,\\
  Ihn zu küßen.\\
  Jeſus aber ſprach zu ihm:

  \voice[Jesus]
  Juda! verrätheſt du des Menſchen Sohn\\
  mit einem Kuß?

  \voice[Evangelist]
  Da aber ſahen, die um Ihn waren,\\
  was da werden wollte, ſprachen ſie zu Ihm:
\end{movement}

\begin{movement}{herrsollen}
  \voice[Chor]
  HErr, ſollen wir mit dem Schwerdt drein ſchlagen?
\end{movement}

\begin{movement}{undeinerausihnen}
  \voice[Evangelist]
  Und einer aus ihnen ſchlug\\
  des Hoheprieſters Knecht\\
  und hieb ihm ſein recht Ohr ab.\\
  Jeſus aber antwortete und ſprach:

  \voice[Jesus]
  Laßet ſie doch ſo ferne machen.

  \voice[Evangelist]
  Und Er rührete ſein Ohr an\\
  und heilete ihn.
\end{movement}

\begin{movement}{lassmichanandern}
  \voice[Chor]
  Laß mich an andern üben,\\
  was du an mir gethan,\\
  und meinen Nächſten lieben,\\
  gern dienen jedermann,\\
  ohn Eigennutz und Heuchelſchein,\\
  und, wie du mir erwieſen,\\
  aus reiner Lieb allein.
\end{movement}

\clearpage
\begin{movement}{jesusaberantwortete}
  \voice[Evangelist]
  Jeſus aber antwortete zu den Hohenprieſtern\\
  und Hauptleuten des Tempels und den Älteſten,\\
  die über Ihn kommen waren:

  \voice[Jesus]
  Ihr ſeid als zu einem Mörder\\
  mit Schwerdtern und mit Stangen ausgegangen.\\
  Ich bin täglich bei euch im Tempel geweſen,\\
  und ihr habt keine Hand an mich geleget.\\
  Aber dies iſt eure Stunde\\
  und die Macht der Finſterniß.

  \voice[Evangelist]
  Sie griffen Ihn aber und führeten Ihn\\
  und brachten Ihn in des Hoheprieſters Haus.\\
  Petrus aber folgete von ferne.\\
  Da zündeten ſie ein Feuer an mitten im Pallaſt,\\
  und ſetzten ſich zuſammen,\\
  und Petrus ſazte ſich unter ſie.\\
  Da ſahe ihn eine Magd ſitzen bey dem Licht\\
  und ſahe eben auf ihn und ſprach zu ihm:

  \voice[Magd I]
  Dieſer Jünger war auch mit Ihm.

  \voice[Evangelist]
  Er aber verleugnete Ihn und ſprach:

  \voice[Petrus]
  Weib, ich kenne Sein nicht.
\end{movement}

\begin{movement}{gepriesnerweibes}
  \voice[Alto]
  Geprißner Weibes Saamen,\\
  ich nenne dich mit Nahmen,\\
  und weiß wohl, wer du biſt.\\
  Du biſt mein Nazarener,\\
  du biſt mein Welt Verſöhner,\\
  des Blut mein höchſtes Labſaal iſt.
\end{movement}

\begin{movement}{unduebereine}
  \voice[Evangelist]
  Und über eine kleine Weile\\
  ſahe ihn eine andere und ſprach:

  \voice[Magd II]
  Du biſt auch deren Einer.

  \voice[Evangelist]
  Petrus aber ſprach:

  \voice[Petrus]\enlargethispage\baselineskip
  Menſch, ich bins nicht.

  \voice[Evangelist]
  Und über eine Weile bey einer Stunde\\
  bekräftigte es eine andere und ſprach:

  \voice[Magd III]
  Warlich, dieſer war auch mit Ihm,\\
  denn er iſt ein Galiläer.

  \voice[Evangelist]
  Petrus aber ſprach:

  \voice[Petrus]
  Menſch, ich weiß nicht, was du ſageſt.

  \voice[Evangelist]
  Und alsbald, da er noch redete,\\
  krähete der Hahn.\\
  Und der HErr wandte ſich und ſahe Petrum an,\\
  und Petrus gedachte an des HErren Wort,\\
  als er zu ihm geſaget hatte:\\
  Ehe denn der Hahn krähet,\\
  wirſt du mich dreymahl verleugnen.\\
  Und Petrus ging hinaus\\
  und weinte bitterlich.

  \voice[Petrus]
  Erbarm es, Gott,
  wo geh ich hin, wo ſoll ich hin?\\
  Ich Armer weiß mir nicht zu rathen.\\
  Dieweil, nach ſo verfluchten Thaten,\\
  in Gott verhaßt und mir zuwieder bin,\\
  vor Angſt erſtarr:\\
  Mit Gram und Sehnen\\
  gedenck ich nun an meine Schuld.\\
  Erſchrocknes Hertz, zerfließe doch in Thränen,\\
  du bringſt dich ſelbſt um Gottes Huld.\\
  Was aber thut die höchſte Liebe,\\
  die jetzt ihr eigen Leyd vergißt,\\
  und um mein Heil bekümmert iſt?\\
  Sie ſpüret, daß ich mich betrübe,\\
  drum blickt ſie mich mitleidig an,\\
  und zeiget mir dadurch,\\
  wie ſie mich lieben kann.
\end{movement}

\begin{movement}{werdetruhig}
  \voice[Petrus]
  Werdet ruhig, ihr Gedanken,\\
  meine Treue [or: Hoffnung] ſoll nicht wanken.\\
  Wer will mich verdammen?\\
  Mein Jeſus iſt hier.\\
  Bin ich mit verlohrnen Schafen\\
  mir zum Schaden eingeſchlafen,\\
  ſo wach ich doch wieder:\\
  Gott würket in mir.
\end{movement}

\begin{movement}{ichfuehlezwar}
  \voice[Chor]
  Ich fühle zwar der Sünden Schuld,\\
  die mich bey dir klagt an,\\
  doch aber deines Sohnes Huld\\
  hat gnug für mich gethan.

  Den ſatz ich dir zum Bürgen ein,\\
  wenn ich ſoll vors Gericht,\\
  ich kann ja nicht verlohren ſeyn\\
  in ſolcher Zuverſicht.
\end{movement}

\begin{movement}{diemaenneraber}
  \voice[Evangelist]
  Die Männer aber, die Jeſum hielten,\\
  verſpotteten Ihn und ſchlugen Ihn,\\
  verdeckten Ihn und ſchlugen Ihn ins Angeſicht\\
  und fragten Ihn und ſprachen:

  \voice[Chor]
  Weisſage, wer iſts, der dich ſchlug?

  \voice[Evangelist]
  Und viel andere Läſterungen\\
  ſagten ſie wider Ihn.\\
  Und als es Tag ward, ſammelten ſich\\
  die Aelteſten des Volcks,\\
  die Hohenprieſter und Schrifftgelehrten,\\
  und führeten Ihn hi[nauf vor ihren] Rath und ſprachen:

  \voice[Chor]
  Biſtu Chriſtus? Sag es uns!

  \voice[Evangelist]
  Er ſprach aber zu ihnen:

  \voice[Jesus]
  Sage ichs euch, ſo gläubet ihrs nicht,\\
  frage ich aber, ſo antwortet ihr mir nicht\\
  und laßet mich doch nicht los.\\
  Darum von nun an wird des Menſchen Sohn\\
  ſitzen zur rechten Hand der Krafft Gottes.

  \voice[Evangelist]
  Da ſprachen ſie alle:

  \voice[Chor]
  Biſtu denn Gottes Sohn?

  \voice[Evangelist]
  Er ſprach zu ihnen:

  \voice[Jesus]
  Ihr ſagets, denn ich bins.

  \voice[Evangelist]
  Sie aber ſprachen:
\end{movement}

\begin{movement}{wasduerfenwir}
  \voice[Chor]
  Was dürfen wir weiter Zeugnis?\\
  Wir habens ſelbſt gehört aus ſeinem Munde.
\end{movement}

\begin{movement}{mundder}
  \voice[Alto]
  Mund der Wahrheit,\\
  deines ewgen Lichtes Klarheit\\
  zeuget von der letzten Zeit\\
  in der tiefſten Niedrigkeit.\\
  Laß uns doch vor dir beſtehen,\\
  wenn wir dich in Wolken ſehen\\
  und des letzten Tages Nacht\\
  deine Läſtrer zitternd macht.
\end{movement}


\part{part2}

\begin{movement}{unddergantze}
  \voice[Evangelist]
  Und der gantze Hauffe ſtund auf\\
  und führten Ihn für Pilatum,\\
  und fingen an, Ihn zu verklagen,\\
  und ſprachen:
\end{movement}

\begin{movement}{diesenfindenwir}
  \voice[Chor]
  Dieſen finden wir, daß Er das Volck abwendet\\
  und verbeut, den Schos dem Kayſer zu geben,\\
  und ſpricht, Er ſey Chriſtus, ein König.
\end{movement}

\begin{movement}{pilatusaber}
  \voice[Evangelist]
  Pilatus aber fragte Ihn und ſprach:

  \voice[Pilatus]
  Biſtu der Jüden König?

  \voice[Evangelist]
  Er antwortete ihm und ſprach:

  \voice[Jesus]
  Du ſageſts.

  \voice[Evangelist]
  Pilatus ſprach zu den Hohenprieſtern und zum Volck:

  \voice[Pilatus]
  Ich finde keine Urſach an dieſem Menſchen.

  \voice[Evangelist]
  Sie aber hielten an und ſprachen:
\end{movement}

\begin{movement}{erhatdasvolck}
  \voice[Chor]
  Er hat das Volck erreget damit,\\
  daß er gelehret hat\\
  hin und her im gantzen jüdiſchen Lande,\\
  und hat in Galiläa angefangen\\
  bis hieher.
\end{movement}

\begin{movement}{daaberpilatus}
  \voice[Evangelist]
  Da aber Pilatus Galiläam hörete,\\
  fragte er, ob Er aus Galiläa wäre.\\
  Und als er vernahm,\\
  daß Er unter Herodes Obrigkeit gehörete,\\
  überſandte er Ihn zu Herodes,\\
  welcher in denſelbigen Tagen\\
  auch zu Jeruſalem war.\\
  Da aber Herodes Jeſum ſahe,\\
  ward er ſehr froh,\\
  denn er hätte Ihn längſt gerne geſehen,\\
  denn er hatte viel von Ihm gehöret\\
  und hoffete, er würde ein Zeichen von Ihm ſehn.\\
  Und er fragte Ihn mancherley.\\
  Er antwortete ihm aber nicht.\\
  Die Hohenprieſter aber und Schriftgelehrten\\
  ſtunden und verklagten Ihn hart.\\
  Aber Herodes mit ſeinem Hofgeſinde\\
  verachtete und verſpottete Ihn,\\
  legte Ihm ein weiß Kleid an,\\
  und ſandte Ihn wieder zu Pilato.\\
  Auf dem Tag wurden Pilatus und Herodes\\
  Freunde miteinander,\\
  denn zuvor waren ſie einander Feind.
\end{movement}

\begin{movement}{istgottfuer}
  \voice[Chor]
  Iſt Gott für mich, ſo trete\\
  gleich alles wider mich.\\
  So oft ich ruf und bete,\\
  weicht alles hinter ſich.\\
  Hab ich das Haupt zum Freunde,\\
  und bin geliebt bey Gott,\\
  was kann mir thun der Feinde\\
  und Widerſacher Spott?
\end{movement}

\begin{movement}{pilatusaberrief}
  \voice[Evangelist]
  Pilatus aber rief die Hohenprieſter\\
  und die Oberſten und das Volck zuſammen\\
  und ſprach zu ihnen:

  \voice[Pilatus]
  Ihr habt dieſen Menſchen zu mir gebracht,\\
  als der das Volck abwende.\\
  Und ſiehe, ich habe Ihn für euch verhöret,\\
  und finde an dem Menſchen der Sachen keine,\\
  deren ihr Ihn beſchuldiget.\\
  Herodes auch nicht,\\
  denn ich habe euch zu ihm geſandt,\\
  und ſiehe, man hat nichts auf Ihm gebracht,\\
  das des Todes werth ſey.\\
  Darum will ich Ihn züchtigen und loß laßen.

  \voice[Evangelist]
  Denn er mußte ihnen einen\\
  nach Gewohnheit des Feſtes los geben.\\
  Da ſchrie der gantze Hauffe und ſprach:
\end{movement}

\begin{movement}{hinwegmitdiesem}
  \voice[Chor]
  Hinweg mit dieſem,\\
  und gib uns Barrabam loß!
\end{movement}

\begin{movement}{welcherwarum}
  \voice[Evangelist]
  Welcher war um einen Aufruhr,\\
  der in der Stadt geſchahe,\\
  und um eines Mords willen\\
  ins Gefängnis geworffen.\\
  Da rief Pilatus abermahl zu ihnen\\
  und wolte Jeſum loßlaßen.\\
  Sie riefen aber und ſprachen:

  \voice[Chor]
  Creutzige Ihn!

  \voice[Evangelist]
  Er aber ſprach zum drittenmal zu ihnen:

  \voice[Pilatus]
  Was hat denn dieſer Übels gethan?\\
  Ich finde keine Urſach des Todes an Ihm.\\
  Darum will ich Ihn züchtigen und los laßen.

  \voice[Evangelist]
  Aber ſie lagen ihn an mit großem Geſchrey\\
  und forderten, daß er gecreutziget würde.\\
  Und ihr und der Hohenprieſter\\
  Geſchrey nahm überhand.\\
  Pilatus aber urtheilete,\\
  daß ihre Bitte geſchehe,\\
  und ließ den los,\\
  der um Aufruhr und Mords willen\\
  war ins Gefängniß geworfen,\\
  um welchen ſie baten.\\
  Aber Jeſum übergab er ihrem Willen.
\end{movement}

\begin{movement}{essollder}\enlargethispage\baselineskip
  \voice[Chor]
  Es ſoll der fromme Jeſus ſterben,\\
  die Raſerey häuft Seine Noth,\\
  man reißt, man führt Ihn in den Todt,\\
  das ſchwere Creutz mit Blut zu färben.
\end{movement}

\begin{movement}{undalssie}
  \voice[Evangelist]
  Und als ſie Ihn hinführeten,\\
  ergriffen ſie einen Simon von Cyrenen,\\
  der kam vom Felde,\\
  und legten das Creutz auf ihn,\\
  daß ers Jeſu nachtrüge.\\
  Es folgeten Ihm aber nach\\
  ein großer Hauffe Volck und Weiber,\\
  die klagten und beweineten Ihn.
\end{movement}

\begin{movement}{essollderb}
  \voice[Chor]
  Es ſoll der fromme Jeſus ſterben,\\
  die Raſerey häuft Seine Noth,\\
  man reißt, man führt Ihn in den Todt,\\
  das ſchwere Creutz mit Blut zu färben.
\end{movement}

\begin{movement}{jesusaberwandte}
  \voice[Evangelist]
  Jeſus aber wandte ſich um zu ihnen\\
  und ſprach:

  \voice[Jesus]
  Ihr Töchter von Jeruſalem,\\
  weinet nicht über mich,\\
  ſondern weinet über euch ſelbſt\\
  und über eure Kinder.\\
  Denn ſiehe, es wird die Zeit kommen,\\
  in welchen man ſagen wird:\\
  ſeelig ſind die Unfruchtbahren\\
  und die Leiber, die nicht gebohren haben,\\
  und die Brüſte, die nicht geſäuget haben.\\
  Denn werden ſie anfahen zu ſagen\\
  zu den Bergen: fallet über uns,\\
  und zu den Hügeln: decket uns.\\
  Denn ſo man das thut am grünen Holtz,\\
  was will am dürren werden?
\end{movement}

\begin{movement}{nimmsichrer}
  \voice[Basso]
  Nimm, ſichrer Menſch, des Höchſten Rath zu Hertzen,\\
  wir müßen nicht mit denen Sünden ſchertzen,\\
  weil dieſer Greul den Eifer Gottes mehrt.\\
  Die Unſchuld hat entſetzlich leiden müßen,\\
  dich aber ſagt dein beißendes Gewißen,\\
  daß dürres Holtz dem Feuer zugehört.
\end{movement}

\clearpage
\begin{movement}{wieheftig}
  \voice[Chor]
  Wie heftig unſre Sünden\\
  den frommen Gott entzünden,\\
  wie Rach und Eifer gehn,\\
  wie grauſam ſeine Ruthen,\\
  wie zornig ſeine Fluthen,\\
  will ich aus dieſen Leiden ſehn.
\end{movement}

\begin{movement}{eswurdenaber}
  \voice[Evangelist]
  Es wurden aber auch hingeführet\\
  zweyen andere Übelthäter,\\
  daß ſie mit Ihm abgethan würden.\\
  Und als ſie kamen an die Städte,\\
  die da heißet Schädel Städte,\\
  creutzigten ſie Ihn daſelbſt\\
  und die Übelthäter mit Ihm,\\
  einen zur Rechten und einen zur Linken.\\
  Jeſus aber ſprach:

  \voice[Jesus]
  Vater, vergib ihnen,\\
  denn ſie wißen nicht, was ſie thun.

  \voice[Evangelist]
  Und ſie theileten ſeine Kleider,\\
  und wurfen das Loos darum.\\
  Und das Volck ſtund und ſahe zu.\\
  Und die Oberſten ſamt ihnen\\
  ſpotteten Sein und ſprachen:
\end{movement}

\begin{movement}{erhatandern}
  \voice[Chor]
  Er hat andern geholfen,\\
  Er helfe Ihm ſelber,\\
  iſt Er Chriſt, der Auserwählte Gottes.

  \voice[Evangelist]
  Es verſpotteten Ihn auch die Kriegesknechte,\\
  traten zu Ihm und brachten Ihn Eßig und ſprachen:

  \voice[Chor]
  Biſt du der Jüden König,\\
  ſo hilf dir ſelber.
\end{movement}

\begin{movement}{verdamlichebosheit}
  \voice[Soprano]
  Verdamliche Bosheit, entſetzliche Wuth,\\
  ſtraf, göttliche Rache, die mördriſche Brut!\\
  Doch ich ſelbſt,\\
  unſchuldigs Lamm,\\
  habe Dich\\
  ans Creutz geſchlagen.\\
  Ich muß mich verdammen zu ewigen Plagen,\\
  mir drohet der Abgrund mit Zittern und Zagen.\\
  Doch, mein Jeſu,\\
  Dein Erbarmen\\
  hilft mir armen,\\
  Du erlöſt mich\\
  durch Dein Blut.
\end{movement}

\begin{movement}{eswarauch}
  \voice[Evangelist]
  Es war auch oben über Ihn geſchrieben\\
  die Überſchrift mit griechiſchen\\
  und lateiniſchen und hebräiſchen Buchſtaben:\\
  Dies iſt der Jüden König.\\
  Aber der Übelthäter einer,\\
  die da gehencket waren,\\
  läſterte Ihn und ſprach:

  \voice[Uebelthäter I]
  Biſt Du Chriſtus, ſo hilf Dir ſelber und uns.

  \voice[Evangelist]
  Da antwortete der andere,\\
  ſtrafte ihn und ſprach:

  \voice[Uebelthäter II]
  Und du fürchteſt dich auch nicht vor Gott,\\
  der du doch in gleicher Verdamniß biſt?\\
  Und zwar wir ſind billig drinnen,\\
  denn wir empfahen was unſer Thaten werth ſind;\\
  dieſer aber hat nichts ungeſchicktes gehandelt.

  \voice[Evangelist]
  Und ſprach zu Ihm:

  \voice[Uebelthäter II]
  HErr, gedencke an mich,\\
  wenn Du in Dein Reich kommeſt.\\
  Evangelist Und Jeſus ſprach zu ihm:

  \voice[Jesus]
  Warlich, ich ſage dir,\\
  heute wirſt du mit mir im Paradieſe ſeyn.
\end{movement}

\begin{movement}{ichbinein}
  \voice[Chor]
  Ich bin ein Glied an Deinem Leib,\\
  des tröſt ich mich von Hertzen.\\
  Von Dir ich ungeſchieden bleib\\
  in Todesnoth und Schmertzen.\\
  Wenn ich gleich ſterb, ſo ſterb ich Dir,\\
  Dein ewig Leben haſt Du mir\\
  durch Deinen Todt erworben.
\end{movement}

\begin{movement}{undeswar}
  \voice[Evangelist]
  Und es war um die ſechſte Stunde,\\
  und es ward eine Finſterniß\\
  über das gantze Land\\
  bis an die neunte Stunde,\\
  und die Sonne verlohr ihren Schein.\\
  Und der Fürhang des Tempels\\
  zuriß mitten entzwey.\\
  Und Jeſus ſchrie laut und ſprach:

  \voice[Jesus]
  Vater, ich befehle meinen Geiſt in deine Hände.

  \voice[Evangelist]
  Und als Er das geſagt,\\
  verſchied Er.
\end{movement}

\begin{movement}{ruhetsanft}
  \voice[Soprano, Alto]
  Ruhet ſanft, ihr holden Glieder,\\
  ſchlafet wohl, es iſt vollbracht.\\
  Chriſtus laſt uns unſre Plagen\\
  mit Gelaßenheit ertragen\\
  bis der frohe Tag erwacht [or: anbricht],\\
  da uns Jeſus ſeelig macht.
\end{movement}

\begin{movement}{daaberder}
  \voice[Evangelist]
  Da aber der Hauptmann ſahe,\\
  was da geſchah,\\
  preiſete er Gott und ſprach:

  \voice[Hauptmann]
  Fürwahr, dieſer iſt ein frommer Menſch geweſen.

  \voice[Evangelist]
  Und alles Volck, das dabey war und zuſahe,\\
  da ſie ſahen, was da geſchah,\\
  ſchlugen ſie an ihre Bruſt\\
  und wandten wieder um.\\
  Es ſtanden aber alle Seine Bekandten von ferne\\
  und die Weiber,\\
  die Ihm aus Galiläa waren nachgefolget,\\
  und ſahen das alles.\\
  Und ſiehe, ein Mann mit Nahmen Joſeph,\\
  ein Rathsherr,\\
  der war ein guter frommer Mann,\\
  der hatte nicht gewilliget\\
  in ihren Rath und Handel,\\
  der war von Arimathia, der Stadt der Jüden,\\
  der auch auf das Reich Gottes wartete.\\
  Der ging zu Pilato und bat um den Leib Jeſu,\\
  und nahm Ihn ab,\\
  wickelte Ihn in ein Leinwandt\\
  und legte Ihn in ein gehauen Grab,\\
  darinnen niemand je gelegen war.\\
  Und es war der Rüſttag und der Sabbath brach an.\\
  Es folgeten aber auch die Weiber nach,\\
  die mit Ihm kommen waren aus Galiläa\\
  und beſchaueten das Grab\\
  und wie Sein Leib geleget ward.\\
  Sie kehreten aber um\\
  und bereiteten die Specerey und Salben,\\
  und den Sabbath über waren ſie ſtille\\
  nach dem Geſetz.
\end{movement}

\begin{movement}{derduherr}
  \voice[Chor]
  Der Du, HErr Jeſu, Ruh und Raſt\\
  in Deinem Grab gehalten haſt,\\
  gib, daß wir in Dir ruhen all\\
  und unſer Leben Dir gefall.

  Verleih, o HErr, uns Stärk und Muth,\\
  die Du erkauft mit Deinem Blut,\\
  und führ uns in des Himmels Licht\\
  zu Deines Vaters Angeſicht.

  Wir danken Dir, o Gotteslamm,\\
  getötet an des Creutzes Stamm,\\
  laß ja uns Sündern Deine Pein\\
  ein Eingang in das Leben ſeyn.
\end{movement}
}

\eesScore

\end{document}
